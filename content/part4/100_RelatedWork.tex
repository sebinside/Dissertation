\chapter{Related Work}%
\label{ch:relatedwork}%
 

In this chapter, we present other work that is related to our contributions\footnote{We are well aware of the advances of generative artificial intelligence in summarizing research articles. Therefore, we assume that handwritten sections on related work---such as this one---could soon become obsolete. Thus, we focus on providing an understandable overview.}.
As the title of this thesis implies, our contributions revolve around three central topics: Uncertainty, software architecture, and confidentiality.
These topics represent the foundations of our work, presented in \autoref{ch:foundations}.
Related work exists in all intersections of these topics.
Thus, we structure related work by iteratively investigating work in these areas.

The remainder of this chapter is structured as follows:
We summarize work from the software engineering and software architecture research community that also addresses uncertainty.
Then, we focus on architecture-based approaches to security and confidentiality.
Last, we investigate uncertainty-aware approaches to confidentiality that do not require architectural abstraction.
This dissertation is at the center of these areas, combining aspects of confidentiality, software architecture, and uncertainty-related research.

\ownpublications{
\fancycite{hahner_dealing_2021}, 
\fancycite{hahner_architectural_2021}, 
\fancycite{boltz_handling_2022},
\fancycite{walter_architectural_2022},\linebreak
\fancycite{hahner_classification_2023},
\fancycite{hahner_model-based_2023},
\fancycite{hahner_architecture-based_2023}, 
\fancycite{camara_uncertainty_2024},
\fancycite{boltz_extensible_2024},
\fancycite{hahner_arcn_2024}
}





\section{Uncertainty and Software Architecture}%
\label{sec:relatedwork:uncertaintyXarchitecture}

The intersection of research on uncertainty and software architecture represents the largest field investigated in this chapter.
This is no surprise, as also the vast majority of our publications were presented at conferences and workshops that belong to these research communities.
Furthermore, uncertainty is often discussed using model-based approaches common in software architectural research \cite{troya_uncertainty_2021,sobhy_evaluation_2021}.
We start by giving an overview of surveys and research roadmaps of both communities.
Afterward, we briefly summarize publications on uncertainty taxonomies and approaches related to \acfp{SAS}.
Then, we summarize the comprehensive work that has been conducted to optimize software architectures or to use architectural knowledge in uncertainty-aware analysis.
Another relation of both areas are design decisions, as already discussed with the \emph{cone of uncertainty} \cite{mcconnell_software_1998}, see \autoref{sec:foundations:uncertainty}.
Last, we summarize work that focusses on uncertainty management and sharing architectural knowledge.


\subsection{Surveys and Research Roadmaps}

Recently, two large surveys were conducted within the \acf{SAS} community that are related to our work.
\textcite{troya_uncertainty_2021} conducted a \acf{SLR} and the modeling of uncertainty.
They investigate 123 primary studies to better understand the notions and formalisms used to represent uncertainty.
Furthermore, they discuss when uncertainty is analyzed and whether such analysis is tool-supported.
They present a comprehensive collection of formalisms used to represent uncertainty, e.g., variability models, fuzzy logic, or temporal logic.
Their results indicate a lack of analysis approaches that are able to handle heterogeneous uncertainty types, such as presented in \autoref{sec:impactanalysis:ufd}.
They also highlight the need for consolidated modeling solutions, as currently driven with the \acf{OMG} \acf{PSUM} \cite{PSUM} standard.
Furthermore, they mention the need of tool-supported processes, which we address with \abunai, \uia, and \arcen.
Regarding design time analyses like the one presented in this thesis, the most common approach are variability models, such as the uncertainty meta model presented in \autoref{sec:confidentialityanalysis:abunai}.
We conclude that our contributions use  techniques well-known from the state of the art, but surpass related work by providing comprehensive modeling support for heterogeneous uncertainty and tool-supported analyses.

\textcite{hezavehi_uncertainty_2021} present an overview of the research community based on questionnaires.
In two stages, they investigate uncertainty concepts, sources, methods, and challenges.
Their results motivate the design time analysis of uncertainty, which includes both uncertainty sources internal and external to the system under study.
They state that \enquote{only when enough knowledge is not available at design time, [\dots], uncertainty handling should be postponed to run-time} \cite{hezavehi_uncertainty_2021}.
Moreover, they find that the current state of the art lacks in including non-functional requirements both as optimization goal as well as side effects.
We address this shortcoming and the problem of dealing with concurrent sources of uncertainty in \autoref{ch:impactanalysis}
By providing a procedure to handle uncertainty in \autoref{sec:overview:procedure} and collecting uncertainty sources in \autoref{sec:classification:collaboration}, we also partially address the challenges of consolidating knowledge, providing guidelines, and dealing with unanticipated change.
Last, the authors present an initial reference process that shows many activities that are also contained in the procedure presented in \autoref{sec:overview:procedure}, e.g., uncertainty identification, modeling, and propagation.
\textcite{calinescu_understanding_2020} provide the findings of another survey in the \acp{SAS} community, focussed on understanding uncertainty.
They also highlight the challenges of dealing with unanticipated change and enhancing the explainability of \acp{SAS}.
With \arcen, we provide a step towards addressing these challenges in \autoref{sec:classification:identification}.
We conclude that our approach fits current research challenges.

Research roadmaps light current research directions and provide agendas and research challenges.
Here, \textcite{de_lemos_software_2013} describe a general roadmap regarding the research on \acp{SAS}.
They especially highlight the challenge of the design space's size.
We discussed this issue and how to address the combinatorial explosion regarding data flow analysis and confidentiality in \autoref{sec:confidentialityanalysis:complexity}.
To also at least partially tackle the identified challenge regarding processes, we describe a procedure with activities and roles in \autoref{sec:overview:procedure}.
More recently, \textcite{weyns_towards_2023} describe their findings on current challenges regarding uncertainty management.
They highlight the need for understanding uncertainty and providing end-to-end approaches.
We present a tailored classification in \autoref{sec:classification:classification}, and a procedure that starts at the identification of uncertainty in \autoref{ch:classification} and ends after analyzing issues due to uncertainty in \autoref{ch:confidentialityanalysis}.
Thereby, we address these challenges regarding confidentiality as a non-functional requirement.


\subsection{Uncertainty Taxonomies and Classifications}

To better understand uncertainty, researchers created several taxonomies \cite{bures_capturing_2020,esfahani_uncertainty_2013,mahdavi-hezavehi_classification_2017,perez-palacin_uncertainties_2014,ramirez_taxonomy_2012,walker_defining_2003,PSUM,camara_uncertainty_2017}.
\textcite{walker_defining_2003} present a taxonomy of uncertainty using three dimensions. 
The \emph{location} describes where the uncertainty can be found, e.g., in the model input or context. 
The \emph{nature} distinguishes between epistemic (i.e., lack of knowledge) and aleatory (i.e., natural variability) uncertainty. 
Last, the \emph{level} describes how much is known about the uncertain influence.
Although this taxonomy has been the baseline for many others, it does not specifically aim to describe software-related uncertainty.
\textcite{perez-palacin_uncertainties_2014} build on this classification in the context of \acp{SAS}. 
They thereby adjust the dimension \emph{location} to better fit software models.
\textcite{bures_capturing_2020} adapt this taxonomy again to \enquote{fit the needs of uncertainty in access control} \cite{bures_capturing_2020}.
Although this work only considers access control in Industry 4.0 scenarios, it is also a good foundation for our classification, presented in \autoref{sec:classification:classification}.
\textcite{esfahani_uncertainty_2013} describe characteristics of uncertainty and hereby focus on the variability and reducibility of different sources of uncertainty.
They highlight the problem of uncertainty in the environment of a software system, e.g., due to the deployment.
\textcite{mahdavi-hezavehi_classification_2017} propose a classification framework of uncertainty.
They aim at architecture-based, \acp{SAS} but do also not consider security, privacy, or confidentiality.
Also related is the uncertainty template by \textcite{ramirez_taxonomy_2012}.
They present a scheme to describe uncertainty sources for dynamically adaptive systems in requirements, design, and runtime.
Due to the different scope, they describe uncertainty in software architecture as \emph{inadequate design} which is not precise enough to identify the impact of software-architectural \cite{hahner_classification_2023} uncertainty on confidentiality.
Still, they also list \emph{unexplored alternatives} and \emph{misinformed trade-off analysis} which motivates our work.
\textcite{armour_five_2000} presents the order of ignorance which also influenced many of the aforementioned taxonomies, e.g., the work of \textcite{perez-palacin_uncertainties_2014}.

More recently, the \ac{OMG} \ac{PSUM} \cite{PSUM} standard represents a joint effort towards the standardization of terminology to describe uncertainty.
However, at the time of writing, this still represents a work in progress.
A more in-depth discussion of related classifications can be found in \autoref{sec:classification:investigating}.
There, we not only show in detail all existing dimensions but also where we adapt or innovate on the state of the art.
This question is also part of our third evaluation goal \goal{3}, presented in \autoref{sec:evaluation:first}.
We conclude that there has been a substantial effort in the last decades to classify uncertainty.
Based on the comprehensive findings of related work we present a novel classification tailored to confidentiality.
Revisiting related work underlines both the need for tailored classification systems and the appropriateness of our classification to confidentiality, see \autoref{sec:classification:investigating} and \autoref{sec:evaluation:first}.


\subsection{Architecting Self-Adaptive Systems}

As discussed in \autoref{sec:foundations:uncertainty} and in \autoref{sec:classification:relation}, uncertainty is central topic in the \acp{SAS} research community.
However, the software architectures considered in this thesis are not self-adaptive, and further research is required to understand the implications of our findings for such software systems completely.
Nevertheless, we briefly summarize related work in this field.

The \emph{Rainbow} framework by \textcite{garlan_rainbow_2004} enables architecture-based self-adaptation.
The architecture layer comprises an adaptation engine and logic for constraint evaluation to monitor the runtime system and to adapt in case of occurring issues, e.g., regarding the system performance.
As this evaluation is model-based, we argue that our approach could be integrated into this framework to simplify its use at runtime.
Proper modeling and analysis support---such as the one presented in this thesis---is required to identify confidentiality violations \cite{seifermann_architectural_2022,schneider_how_2024}.
In another publication, the authors present the \emph{Znn.com} system to evaluate the Rainbow framework.
We build on this software system to present our approach to addressing the \ac{UIP} in \autoref{sec:impactanalysis:ufd}.

\textcite{andersson_modeling_2009} present modeling dimensions of \acp{SAS}, i.e., variation points of said systems.
They distinguish four groups of dimensions: Goals, changes, mechanisms, and effects.
In the scope of this work and the already discussed uncertainty classifications, this can be seen as a more general description of the underlying challenges of \acp{SAS}.
Using our terminology, uncertainties causing changes have sources and types, as described in \autoref{sec:classification:relation}.
Here, our classification describes a small subset of these dimensions in more detail, justified by the purpose of conducting confidentiality analysis.
This underlines the relation of our work to \acp{SAS}, as already illustrated in recent publications \cite{weyns_towards_2023,camara_uncertainty_2024}.

\textcite{whittle_relax_2009} present \emph{RELAX}, an approach targeted at considering uncertainty in requirements engineering of \acp{SAS}.
By weakening requirements depending on the environmental conditions monitored during runtime, self-adaptation can be achieved in a structured manner.
The authors argue that, compared to ad-hoc approaches, considering uncertainty in requirements helps to describe clear adaptation boundaries and to ensure invariants.
However, this approach does not consider the development of the system nor provides mechanisms to satisfy the weakened requirements. 
In addition, it can only cope with environmental uncertainty that is known during formulating the requirements.
Nevertheless, we argue that revisiting confidentiality requirements under uncertainty represents a promising research direction for future work.


\subsection{Architecture Evaluation Under Uncertainty}

The topic of architecture evaluation under uncertainty represents the most comprehensive section of related work.
Numerous approaches \cite{esfahani_dealing_2012,koziolek_peropteryx_2011,aleti_archeopterix_2009,lytra_supporting_2013,vanherpen_design-space_2014,esfahani_guidearch_2013} have been proposed to evaluate and optimize software architectures.
They often refer to the variability or inherent uncertainty in software architecture as \emph{design space exploration}, which can also be described as \enquote{searching better or even optimal designs} \cite{koziolek_automated_2011}.
\textcite{sobhy_evaluation_2021} conducted a \ac{SLR} comprising 48 primary studies on this topic.
They highlight the focus on design time, which matches our approach.
We summarize the most related approaches.

\emph{PerOpteryx} \cite{koziolek_peropteryx_2011,martens_automatically_2010,koziolek_automated_2011} is a model-based approach to optimize software architectures regarding quality properties like performance, cost, or reliability.
\textcite{koziolek_peropteryx_2011} employ evolutionary optimization to find Pareto-optimal architectural candidates.
To assess the quality of the candidates, they use the Palladio approach, introduced in \autoref{sec:foundations:architecture}.
However, there is no support for confidentiality, although there is already a PerOpteryx extension for security \cite{busch_modelling_2016}.
In contrast to our approach, they modeled these by a concept of concerns. 
These concerns describe which design decisions are dependent on each other. 
In our analysis we focus on the direct impact on confidentiality.
Similarly to the relation to Rainbow \cite{garlan_rainbow_2004}, we argue that our approach can be incorporated or combined with PerOpteryx.
We demonstrated this while defining a scenario-aware data flow analysis regarding structural uncertainty in \autoref{sec:confidentialityanalysis:typespecifc}.

\textcite{esfahani_guidearch_2013} present \emph{GuideArch}, an approach to explore the architectural solution space under uncertainty. 
This shall enable software architects to identify critical design decisions.
They apply fuzzy math to represent uncertainty and its impact on the software architecture.
This enables the comparison of architectural candidates and the exploration of the solution space.
Although this approach considers uncertainty and fuzziness on the architectural abstraction level, it does not consider confidentiality or other privacy-related quality properties.
By applying fuzzy methods in graph-aware data flow analysis, we found that their expressiveness is limited regarding confidentiality, which is hard to quantify, see \autoref{sec:confidentialityanalysis:typespecifc}.
Thus, we argue that the application of GuideArch to analyze confidentiality is not straightforward.

Another design space exploration tool is \emph{ArcheOpterix} by \textcite{aleti_archeopterix_2009}. 
It can also optimize a given architecture for multiple criteria using evolution techniques and design constraints.
However, similarly, it does not support a confidentiality analysis, and the design space modeling is more restrictive.
The quality attributes of \enquote{safety, reliability, security, performance, timeliness, and resource consumption} \cite{aleti_archeopterix_2009} are only considered to be future work for ArcheOpterix.
At the time of writing, we were unable to find analysis extensions suitable for security, privacy, or confidentiality.
\textcite{vanherpen_design-space_2014} also present an approach to model-based design space exploration.
They present a pattern catalog of techniques known from architecture optimization but do not focus on analysis support.
\textcite{gerasimou_synthesis_2018} present \emph{EvoChecker}, a search-based approach employing evolutionary algorithms in automated model synthesis.
They focus on quality of service properties like reliability, response time, and cost.
In sum, design space exploration approaches are versatile, but do often not support security-related attributes but focus, e.g., on performance.
While these approaches can analyze a wide variety of quality properties, they are not appropriate to consider confidentiality as they lack the required expressiveness to consider data processing and data flow constraints.

Last, there is a relation to the topics of variability, product lines, and testing, as they share similar challenges regarding the size of the design space and combinatorial explosion, as discussed in \autoref{sec:confidentialityanalysis:complexity}.
Here, \textcite{abbas_modeling_2012} propose to model the variability in quality concerns, thereby expressing what does vary how and for which reason.
This shall reduce the number of required quality attribute scenarios.
Another approach is the combination of design models and test models to minimize testing efforts while facing feature interaction and ensuring a comprehensive coverage of interactions.
\textcite{oster_pairwise_2011} combine pairwise feature generation with model-based testing and \textcite{lochau_model-based_2012} combine feature models and state charts that represent test models.
The challenge of feature interactions is comparable to dealing with uncertainty interactions, see \autoref{sec:impactanalysis:ufd}.
By employing the notion of independent \acfp{TFG}, we can minimize the required analysis effort due to interacting uncertainties, see \autoref{sec:confidentialityanalysis:framework}.
Here, mapping the uncertainty model to the architectural model to minimize the analysis effort is comparable to the mapping applied in related work.
We require full coverage of all interactions on a single \ac{TFG}, as a missing variation could cause missed confidentiality violations due to uncertainty, see the discussion of recall in \autoref{sec:evaluation:overview}.


\subsection{Architecture-Based Analysis}

Besides automatically optimizing software architectures proposed in the previous paragraphs, architectural models are often used as a means to analyze the quality of software architectures or to help in the design process.
These approaches are closely related to this thesis as we also build on the architectural model to analyze confidentiality.
In the following, we summarize related architecture-based analyses.

When considering architectural models as a baseline for uncertainty analysis, we first have to address the question of incorporating uncertainty into the model.
Here, \textcite{garlan_software_2010} proposed already more than a decade ago to consider uncertainty as a first-class entity.
There are numerous notations for incorporating uncertainties of different types into software models \cite{troya_uncertainty_2021}. 
For instance, SysML~\cite{object_management_group_omg_2023} and the \acf{UML} MARTE Profile~\cite{object_management_group_uml_2011} provide stereotypes and properties to represent some types of uncertainty, especially measurement uncertainty. 
However, existing notations allow modeling mostly homogeneous uncertainties, i.e., of the same type.
The aforementioned \ac{PSUM} standard \cite{PSUM} also provides a metamodel for representing different types of uncertainty but is still a work in progress.
Regarding measurement uncertainty, \textcite{bertoa_incorporating_2020} propose its inclusion into primitive data types of \ac{UML} models.
Our modeling approach of uncertainty and our distinction between five central uncertainty types is also inspired by the aforementioned work, see \autoref{sec:classification:investigating}.
As discussed in \autoref{sec:classification:classification} and \autoref{sec:confidentialityanalysis:typeagnostic}, we can express all identified uncertainty sources \cite{hahner_arcn_2024} as part of the architectural model.

\textcite{famelis_managing_2019} introduce \emph{DeTUM}, which stands for design time uncertainty management.
This tool-supported approach handles uncertainty using partial models.
It introduces uncertainty in the start phase and then resolves it in later stages.
They also refer to open design decisions as a source of uncertainty, similarly to the \emph{cone of uncertainty} \cite{mcconnell_software_1998}, see \autoref{sec:classification:relation}.
However, the authors do not mention security-related quality attributes like confidentiality.
They explicitly mention the requirement of external support to assess the impact of uncertainty---as we offer it with our work.

\textcite{goseva-popstojanova_assessing_2003} present an approach for architecture-based reliability analysis under uncertainty.
Here, values such as the probability that a particular component is used or the reliability of a component contribute to the analysis of uncertainty.
Using Monte Carlo simulation, the authors can analyze how uncertainty propagates from model parameters into reliability estimations.
Although we investigate another quality property and use data flow analysis instead Monte Carlo simulation, we argue that our method is comparable.
Both approaches use appropriate architectural models and derive the impact of uncertainty.
However, we argue that confidentiality and reliability as quality properties differ fundamentally.
Therefore, the approach is not suitable for detecting confidentiality breaches, and we are not able to analyze reliability.
In sum, there exist many approaches to model and analyze uncertainty regarding other quality properties of software architectures. 
However, they do not explicitly take confidentiality into account.


\subsection{Architectural Design Decisions and Uncertainty}
\acfp{ADD} are related to uncertainty.
As depicted with the \emph{cone of uncertainty} \cite{mcconnell_software_1998}, open design decisions introduce uncertainty about the software system to design, and making decisions reduces said uncertainty.

\textcite{kruchten_ontology_2004} presents an ontology of \acp{ADD}.
The author distinguishes between existence, property and executive decisions and provides an overview of \acp{ADD} attributes.
This is especially relevant when considering uncertainty that can void existing decisions and require software architects to backtrack.
\textcite{jansen_software_2005} see software architecture as a composition of \acp{ADD}.
This shows how uncertainty, e.g., about the system context, can hinder good software design as the \emph{best} decision might not be found.
Although both approaches do not focus on uncertainty, they inspired our classification which is strongly coupled to architectural design, see \autoref{sec:classification:classification}.

\textcite{lytra_supporting_2013} propose the use of fuzzy logic to incorporate inherent uncertainty into reusable \acp{ADD}.
This shall enable software architects to share and reuse knowledge about the impact of uncertainty on quality attributes.
They employ a \acf{FIS} to identify the most appropriate design decisions and also foster the importance of decision documentation.
This is related to our data flow analysis under environmental uncertainty that is also based on fuzzy inference, see \autoref{sec:confidentialityanalysis:typespecifc}.
Furthermore, we also address knowledge exchange and documentation with our uncertainty catalog, see \autoref{sec:classification:collaboration}.
Although this approach can also handle security-related quality attributes, violations due to integration issues remain hidden.
Here, the integration of a dedicated analysis---such as our confidentiality analysis---helps identifying also fine-grained problems.
Future work could investigate whether defining reusable \acp{ADD} to ensure confidentiality is feasible.

\textcite{zimmermann_reusable_2007} present a framework to support the identification, making, and enforcement of \acp{ADD}.
This is especially helpful regarding the aforementioned backtracking, i.e., reverting decisions in case of design errors.
Here, awareness of dependencies minimizes the required effort and the chance of introducing new errors while backtracking.
\textcite{noppen_jar_software_2008} discuss design decisions under imperfect information by explicitly modeling uncertain aspects of the architecture based on fuzzy techniques and design trees to record the design history.
\textcite{busari_radar_2017} present the \emph{RADAR} approach \cite{busari_modelling_2019} that comprises a modeling language and decision analysis support.
They highlight dependencies between decisions and provide support to find Pareto-optimal solutions.
Although these approaches help to minimize errors during the design and help to recall the decision paths, they do not support analyzing software systems regarding confidentiality.
Nevertheless, we argue that such frameworks are helpful in combination with modeling and analysis support, which should be investigated in the future.


\subsection{Uncertainty Management and Knowledge Sharing}

Regarding the management of uncertainty and \acp{ADD}, the previous paragraphs introduced a multitude of approaches for modeling, analyzing, and optimizing software architectures.
In the following, we focus on uncertainty management and knowledge sharing beyond the design of single software systems.
Collecting and consolidating knowledge on architectural design and uncertainty supports software architects \cite{weyns_towards_2023,lytra_supporting_2013,zimmermann_reusable_2007,gerdes_decision_2015}.
It can help to address unanticipated change that is not completely unforeseen and can be tackled, e.g., by minimizing assumptions or creating reusable system building blocks. \cite{garlan_unknown_2021}.
The need for uncertainty management, reusable methods, and end-to-end approaches has also been highlighted in the aforementioned surveys and research roadmaps \cite{weyns_towards_2023,weyns_introduction_2020,troya_uncertainty_2021,hezavehi_uncertainty_2021}.
Knowledge sharing to counter uncertainty is also known from legal sciences \cite{sterz_intelligente_2022}.

With \arcen, we present a catalog of uncertainty sources to overcome the limitation of knowledge being scattered among researchers and institutions.
A related approach is \emph{Decision Buddy}, a tool-supported collaborative approach regarding design decisions by \textcite{gerdes_decision_2015}.
By collecting and describing the effects of \acp{ADD}, the decision-making of software architects can be supported with suitable and ranked solutions.
\emph{privacypatterns.eu} is a web-based collection of patterns to enhance privacy \textcite{colesky_system_2018}.
The collection comprises patterns like access control, single point of contact, or informed consent \cite{colesky_system_2018-1}
Last, \emph{arc42} represents an interactive collection of software qualities \cite{starke_arc42_2024}, comprising requirements, definitions, and relations.
These works provide comparable approaches to our proposed solution but cannot be applied to uncertainty and confidentiality due to the domain gap.
Additionally, the aforementioned tooling is partially fragile due to link decay \cite{goh_link_2007} or lacks open accessible data.
To counter such issues and to reach high usability and longevity, our tool support is publicly available, see \autoref{sec:classification:collaboration}.
Ultimately, we want to stress the importance of efforts towards catalogs and guidelines, especially with regard to cooperation among researchers and science transfer\footnote{The importance of guidelines and checklists was already emphasized by practitioners at EMLS' 21, in the very first workshop in which our approach was discussed \cite{hahner_dealing_2021}. \arcen represents one result of these findings, which received very positive reviews and feedback from the community.}.

\textcite{jasser_reusing_2016} present a repository of security solutions to support \acp{ADD}.
Exemplary solutions are access control, input sanitization, or the principle of the least privilege.
Here, the focus on security as a central quality property enables more precise classification, e.g., by relating to sub-goals like confidentiality, integrity, or availability \cite{international_organization_for_standardization_isoiec_2018}.
We consider this to be highly related to our catalog approach, although we approach the challenge from the opposite direction: Instead of providing solution techniques, we focus on supporting the identification of potential problems in the form of uncertainties.

Last, \textcite{lupafya_framework_2022} present a framework for considering uncertainty in the software design.
Their conceptual model connects viewpoints to uncertainties, risks and opportunities, and their mitigation or exploitation.
Furthermore, they present a classification of uncertainty based on consolidating existing work.
Although the approach looks promising at first sight, it falls short of providing further management, modeling, or analysis support beyond describing uncertainty sources with attributes.
Furthermore, combining existing classifications without further purpose lacks novelty \cite{kaplan_introducing_2022}.


\subsection{Summary}

The intersection of research on software architecture and uncertainty presented in this section represents the largest body of related work.
According to the aforementioned surveys, the state of the art is still of an exploratory nature, lacks comprehensive tool support \cite{troya_uncertainty_2021,hezavehi_uncertainty_2021}, and does also not focus on confidentiality.
Our approach addresses this gap regarding confidentiality with ready-to-use tool support that can easily be incorporated into existing approaches.
Furthermore, related work often inspired us.
For instance, our classification is based on existing taxonomies but tailored to confidentiality.
The decision to opt for a web-based and open-source catalog approach was also influenced by the perceived link decay \cite{goh_link_2007} of related work.
Our approach is by far not the first to incorporate uncertainty modeling in the form of scenarios into the architectural design.
However, we were unable to find related work that can identify confidentiality violations due to uncertainty.
We conclude that our approach represents novel work that can be combined with existing frameworks to compose comprehensive end-to-end approaches \cite{weyns_introduction_2020}.

\finding{Related work provides the foundation and inspiration for many aspects of our research.
However, to the best of our knowledge, there exists no directly comparable approach.
Ultimately, combining multiple analyses that differ in supported uncertainty sources and targeted quality properties is expedient.}





\section{Software Architecture and Confidentiality}%
\label{sec:relatedwork:architectureXconfidentiality}

The next section of related work we discuss in this thesis represents the intersection of software architecture and confidentiality.
We present work from the field of model-based security analysis and existing approaches to architectural data flow analysis.
Afterward, we revisit the specification of confidentiality requirements as they represent one important input to data flow analysis that can impact the accuracy of identified confidentiality violations.
This includes related work on access control.


\subsection{Model-based Security Analysis}

Model-based analyses use the model representation of software systems to evaluate their quality, e.g., by using architectural models.
They use techniques known from \acf{MDSD}, see \autoref{sec:foundations:mds}.
\textcite{nguyen_extensive_2015} present a \ac{SLR} based on 108 primary studies.
They find that the vast majority of papers use standardized notations like the \ac{UML}, but only less than half of the investigated papers consider confidentiality.
When confidentiality is discussed, it is often combined with considering authentication or authorization.
Only 11\% of the investigated publications focus on analyzing confidentiality.
They also find a lack of analyses that simultaneously consider multiple security concerns.

\textcite{goos_umlsec_2002} present \emph{UMLsec} that extends \ac{UML} by defining a security profile \cite{juerjens_principles_2002,goos_towards_2001}. 
By incorporating security-related information into \ac{UML}, software architects can reuse existing \ac{UML} diagrams for security analysis.
It supports different kinds of analyses, such as information flow or access control. 
In contrast to our confidentiality analysis approaches, it does not support access control on data, or more complex confidentiality requirements that go beyond, for instance, encrypted communication.
Another \ac{UML} security profile is \emph{SecureUML} by \cite{lodderstedt_secureuml_2002}. 
It supports \acf{RBAC} together with statements in the \acf{OCL} to support dynamic properties in authorization constraints.
Similarly, there is no support for more advanced constraints regarding confidentiality.
Although both approaches represent relevant steps towards model-driven security, they both show the lack of comprehensive support for confidentiality beyond access control, or encryption, as already discussed in the aforementioned survey.
Furthermore, they do not support uncertainty in their analyses.
Last, \textcite{ronneberg_quantitative_2024} propose information flow analysis in \ac{CBSE}, following the approach of correctness-by-construction to enable security guarantees.

\textcite{walter_architectural_2022-1} present a comprehensive approach to identifying confidentiality violations using attacker propagation \cite{walter_architectural_2022-1,walter_architecture-based_2023} and attack path analysis \cite{walter_architecture-based_2023-1,walter_context-based_2023}.
They extend the \acf{PCM} \cite{reussner_modeling_2016} to express vulnerabilities and access control with the \acf{XACML} \cite{oasis_extensible_2013}.
This enables the propagation of attackers within the software architecture.
After each propagation step, the analysis considers which credentials or access rights the attacker might have obtained, e.g., due to vulnerabilities within the system.
The approach also considers dependencies within the software architecture.
For instance, if an attacker gains access to a resource container, this also affects all components deployed on this resource.
The resulting attack graphs shall help software architects to estimate the impact of vulnerabilities and access control decisions.
This approach is highly related to our work, as it uses a similar approach to architectural models \cite{reussner_modeling_2016} and architecture-based propagation \cite{hahner_architecture-based_2024,busch_architecture-based_2020}.
However, it has two limitations regarding our needs.
First, there is no consideration of data flows, as confidentiality violations are identified by tracing attack paths.
Second, there is no comprehensive support for considering uncertainty, although initial approaches exist \cite{walter_architecture-based_2023}.
The authors acknowledge this limitation and state that \enquote{the combination with other uncertainty mitigation approaches} \cite{walter_context-based_2023} could be expedient.


\subsection{Data Flow Analysis}

The concept of \acp{DFD} and data flow analysis is not new \cite{demarco_structure_1979}.
Nevertheless, data flow analysis approaches are often used to assess the security of software systems.
This can be achieved by building on noninterference like \emph{JOANA} \cite{snelting_checking_2014} or deductive verification like \emph{KeY} \cite{ahrendt_deductive_2016}.
Other approaches extract data flows from source code without considering the software architecture.
\emph{RogueOne} by \textcite{sofaer_rogueone_2024} uses data flows to identify rogue updates, i.e., malicious changes to widely used libraries that aim to attack dependent software systems.
More recently, GitHub \cite{github_codeql_2021} pushes the adoption of \emph{CodeQL} \cite{de_moor_ql_2008} for data flow-based source code analysis and vulnerability detection.
Due to the broad applicability, such analyses represent promising approaches to analyzing the confidentiality of real-world software.
However, they lack the connection to the architectural abstraction, which becomes visible, e.g., when considering architecture-related information like the deployment.
Here, approaches to connect source code analyses and architectural analyses exist, e.g., by \textcite{kramer_model-driven_2017} and \textcite{reiche_modeling_2021}.
However, they do not consider uncertainty.
Combining source code analyses with architectural analyses and making uncertainty-aware represents a promising direction for future research.
With appropriate tool support, \acp{DFD} represent a powerful and commonly used mechanism for threat analysis \cite{bernsmed_adopting_2022} that helps in identifying security-related issues \cite{schneider_how_2024}.

The most related architectural data flow analysis is presented by \textcite{seifermann_detecting_2022}.
Their approach considers additional context information, such as the deployment, enabling software architects to analyze confidentiality during early design phases. 
We introduced their underlying meta model for the confidentiality-focused analysis of \acp{DFD} in \autoref{sec:foundations:dfd}.
Furthermore, they present an approach for label propagation to identify confidentiality violations.
Software architects can specify and analyze confidentiality requirements regarding noninterference, encryption, or access control.
A \acf{DSL} \cite{hahner_modeling_2021,hahner_domain-specific_2020} enables the specification of data flow constraints.
Our data flow analysis framework \cite{boltz_extensible_2024}, introduced in \autoref{sec:confidentialityanalysis:framework}, builds on this approach. 
However, the original data flow analysis is unable to consider uncertainty.
The authors acknowledge this limitation and the importance of considering uncertainty in the architectural design and analysis \cite{seifermann_architectural_2022}.

\textcite{boltz_model-based_2022} focus on the collaboration of legal and software experts in the assessment of information security and data protection.
They propose a model-based approach using the data flow analysis framework \cite{boltz_extensible_2024} described in \autoref{sec:confidentialityanalysis:framework}.
By using the \acf{ADL} \ac{PCM}, analyses of multiple quality properties are also possible, e.g., regarding performance and security \cite{boltz_modeling_2024}.
However, their approach only considers uncertainty coming from the legal side.
Here, combining our findings on modeling and analyzing uncertainty could be a promising approach.
\textcite{pilipchuk_architectural_2021} present an access control analysis based on business processes to align processes and access control policies \cite{pilipchuk_aligning_2018}.
By extracting access control requirements from business processes and using them in architecture-based data flow analysis, forbidden data flows can be identified.
However, this approach also focuses on conformance without considering uncertainty.

\textcite{peldszus_secure_2019} also present a data flow analysis approach called \emph{SecDFD}.
They check for compliance between models of the design and the implementation to identify violations.
The mapping is automated but lacks support for custom analysis definitions or data flow constraints and also does not consider uncertainty.
The early detection of design flaws \cite{tuma_automating_2020,tuma_flaws_2019} is closely related to the aforementioned data flow analysis by \textcite{seifermann_detecting_2022} and was also a baseline for the unified modeling primitives for \acp{DFD} used in our work, see \autoref{sec:foundations:dfd}.
Another approach is \emph{iFlow} by \textcite{katkalov_model-driven_2013}.
They use \ac{UML} models to derive and analyze data flows.
Furthermore, the generated source code can be verified.
\textcite{gerking_model_2018} present a model-based information flow analysis to identify timing channels.
Both approaches do not consider uncertainty, but \textcite{gerking_model-driven_2020} mentions the incorporation of uncertainty as potential future work.

Last, \textcite{schneider_automatic_2023} present an automated approach to extract \acp{DFD} from the source code of microservice applications.
They enrich the extracted \acp{DFD} with security-related information, e.g., regarding data storage, passwords, or logging.
This enables comprehensive data flow analysis that also considers architectural information.
They also published a large data set comprising the \acp{DFD} of 17 real-world microservice systems \cite{schneider_microsecend_2023}.
In a proof of concept, we transformed and analyzed this data set using our data flow analysis framework \cite{boltz_extensible_2024} to replicate the identified confidentiality violations.
There are also other steps towards security benchmarks \cite{bambhore_tukaram_towards_2022}.
We state that this represents a promising first step towards comprehensive data flow analysis that considers both the information from the source code and the software architecture.
However, similar to the previously discussed approaches, uncertainty is not considered. 


\subsection{Modeling Confidentiality Requirements}

The data flow analysis approaches presented in this thesis use confidentiality requirements in the form of data flow constraints as input, see \autoref{sec:foundations:dfd}.
Similarly, many related approaches discussed in the previous paragraphs use a dedicated requirement specification language.
Often, confidentiality requirements are also specified using access control \cite{nguyen_extensive_2015}.
The specification of confidentiality requirements impacts the identified confidentiality violations, with our without considering uncertainty.
We briefly summarize related work in both fields in the following.

\textcite{onabajo_properties_2006} investigated the properties of confidentiality requirements using grounded theory.
They present a model for requirements comprising elements like data, stakeholders, statements, purpose, and temporal validity.
This shall support the precise definition of requirements and enable the formal reasoning on derived rules, e.g., from legal frameworks like the \acf{GDPR} \cite{council_of_european_union_regulation_2016}.
To provide a foundation to describe such confidentiality requirements for data flow analysis, we presented a meta model and a \ac{DSL} in previous work \cite{hahner_modeling_2021}.
Data flow constraints follow the pattern of disallowing flows under certain conditions based on the characteristics of the data and involved entities or parts of the system.
A generalized form of this \ac{DSL} was presented with the data flow analysis framework \cite{boltz_extensible_2024}.
However, these approaches are not able to express restrictions with regard to uncertainty, e.g., whether certain flows shall not be allowed under given uncertain conditions.
Here, more research is required.

As discussed previously, access control is often used to express confidentiality requirements through access control policies \cite{seifermann_architectural_2022}.
To guide software architects from high-level requirements to low-level policies, access control policy refinement techniques have been proposed \cite{yang_security_2013}.
Su et al. \cite{linying_su_automated_2005} discuss the automated decomposition of policies based on the resource hierarchy in distributed applications.
\textcite{he_requirements-based_2009} present an approach to define and refine access control policies by analyzing the specification of requirements and the system's database design.
Furthermore, both model-based \cite{massacci_model-driven_2008,maeder_modeling_2020} and verification-based \cite{mery_specication_2007,cheminod_comprehensive_2019} approaches exist, e.g., using logic programming languages like Prolog \cite{craven_policy_2011,zhao_policy_2011}.
Unfortunately, all of these approaches do not explicitly consider any kind of uncertainty.
The relation of uncertainty, i.e., the lack of knowledge and the specification and refinement of access control policies that respect uncertainty represents an interesting direction for future work.


\subsection{Summary}

The intersection of research on software architecture and confidentiality shows a multitude of modeling and analysis approaches.
Following the principle of security and privacy by design \cite{schaar_privacy_2010}, the early consideration and analysis of security and privacy is expedient.
Moreover, fixing issues in later phases is usually more costly \cite{shull_what_2002,boehm_software_2001}.
Related work often uses a model representation to perform security analysis, e.g., based on \ac{UML}, \acp{DFD}, or the \ac{PCM}.
Although we identified many promising approaches, we were unable to identify an architecture-based analysis that both considers confidentiality and uncertainty.
Many approaches acknowledged this limitation \cite{walter_context-based_2023,seifermann_architectural_2022,gerking_model-driven_2020}.
Nevertheless, especially our data flow analysis framework \cite{boltz_extensible_2024} is inspired by related work regarding architecture-based \cite{seifermann_architectural_2022} and code-based \cite{schneider_automatic_2023} data flow analysis.
We repeat the conclusion of the previous section that these findings support the novelty of our work.
Moreover, analysis combinations represent promising approaches, e.g., by closing the gap between the design and the implementation regarding confidentiality analysis under uncertainty.

\finding{Related work shows a multitude of security analysis approaches supporting confidentiality based on models specified using the \ac{UML}, \ac{PCM}, or \acp{DFD}.
Despite their versatility, they lack support for considering uncertainty within the modeling and analysis of confidentiality.}





\section{Confidentiality and Uncertainty}%
\label{sec:relatedwork:confidentialityXuncertainty}

The last section of related work discussed in this chapter is the intersection of research on confidentiality and uncertainty.
This represents the smallest part of related work as we were unable to identify comprehensive approaches that consider confidentiality under uncertainty.
The aforementioned surveys \cite{troya_uncertainty_2021,sobhy_evaluation_2021} support this shortcoming, as only a few identified approaches explicitly target security; confidentiality is not mentioned at all.
We split related work into two groups: uncertainty-aware confidentiality analysis and access control under uncertainty to ensure confidentiality.


\subsection{Uncertainty-Aware Confidentiality Analysis}

We identified two approaches to confidentiality analysis under uncertainty.
First, work regarding uncertainty in cloud computing and, second, work regarding risk assessment of data breaches.
\textcite{tchernykh_towards_2019} discuss different types of uncertainty in the context of cloud computing.
Moreover, they propose approaches to minimize the impact of uncertainty on the systems' reliability and privacy, e.g., data replication, error correction, and homomorphic encryption.
However, the presented work is still in a preliminary state and lacks modeling or automated analysis.

\textcite{morali_it_2008} focus on the assessment of risk regarding confidentiality.
By propagating data breaches through the modeled software system, the criticality can be measured.
The software system is modeled as a graph showing the dependencies between infrastructure nodes regarding exchanged information.
Edges of this graph are annotated with the propagation likelihood of attackers, which enables the modeling of risk.
However, it is not specified how software architects should be able to annotate every edge with an appropriate propagation likelihood and the evaluation only comprises a single case study.
The authors state further studies as future work.
Although we assume that also this approach is still preliminary, the modeling of data dependencies and the propagation of attackers is promising.
The analysis is related to the work to \textcite{walter_architecture-based_2023-1} but requires high modeling effort and does also not claim any automation.


\subsection{Access Control Under Uncertainty}

As discussed previously, access control represents a common approach to ensure the confidentiality of data in software systems.
Here, many approaches to access control under uncertainty exist.
Although these approaches are not directly comparable to our approach of architecture-based modeling and analysis of confidentiality under uncertainty, they comprise related concepts like fuzziness and taking known uncertainty into account.  

\textcite{bures_capturing_2020} discuss the relation of uncertainty and access control in highly dynamic environments like Industry 4.0.
Here, uncertainty-aware access control policies directly consider imperfect information in modeling access decisions.
\textcite{hengartner_distributed_2007} present an access control model for distributed systems that incorporates trust by explicitly specifying remaining uncertainty in access decisions. 
Other approaches regarding uncertainty in access control also utilize fuzzy logic, e.g., to represent security patterns \cite{hosmer_using_1992} or to create risk-adaptive access control models to cope with the uncertainty \cite{cheng_fuzzy_2007}. 
Numerous other approaches also discuss fuzzy approaches to access control \cite{ni_risk-based_2010,martinez-garcia_fuzzy_2011,salim_approach_2011,molloy_risk-based_2012,mahalle_fuzzy_2013,santos_dynamic_2014}.
Only a few take the described \emph{known uncertainty} into account when making access decisions \cite{ardagna_supporting_2006,cuppens_modelling_2003}.

However, details about such policies are usually not specified in the architectural abstraction but are added during policy refinement. 
Here, the high degree of uncertainty regarding the structure, behavior, and usage of the software does not allow one to draw precise conclusions on the confidentiality of the overall system. 
\textcite{cheng_fuzzy_2007} explain this problem with unforeseeable tradeoffs while defining policies. 
Strict policies may reduce the risk of data breaches but may harm the flexibility of software systems, especially in highly dynamic environments like implied by Industry 4.0 \cite{bures_capturing_2020}. 
The common gap of uncertainty-aware approaches to model access control is the lack of refinement of high-level confidentiality requirements whose abstraction is also a source of uncertainty.
In sum, uncertainty-aware access control can be seen as an alternative approach to deal with uncertainty in a software system \cite{perez-palacin_dealing_2014}.


\subsection{Summary}

As discussed previously, the intersection of confidentiality and uncertainty yields the least amount of related research.
One reason could be the focus of the research community on other qualities like performance and reliability \cite{hezavehi_uncertainty_2021,sobhy_evaluation_2021}.
Nevertheless, we identified related uncertainty-aware confidentiality analyses and numerous approaches to access control under uncertainty.
Both fall in one of the two categories of handling uncertainty, discussed by \textcite{perez-palacin_dealing_2014}.
The model can be refined to incorporate the uncertainty and become more resilient, like the approach to consider uncertainty in cloud computing by \textcite{tchernykh_towards_2019}.
Otherwise, the uncertainty can be actively managed as part of the model, e.g., in fuzzy access control policies \cite{cheng_fuzzy_2007,ni_risk-based_2010,martinez-garcia_fuzzy_2011}.

\finding{Related work comprises only a few approaches to uncertainty-aware confidentiality analysis.
The identified approaches are of a preliminary nature and lack comprehensive modeling and automated analysis support.}





\section{Summary and Outlook}%
\label{sec:relatedwork:summary}

In this chapter, we provided an overview of the related work.
First, we focused on work from the areas of software architecture and uncertainty.
Here, many surveys like \acp{SLR} and roadmaps have been discussed in the last decades.
We investigated taxonomies related to our classification, introduced in \autoref{ch:classification}, and architecture-based analyses related to our impact analysis, introduced in \autoref{ch:impactanalysis}.
Furthermore, we summarized the comprehensive state of the art in architecture evaluation that is related to our data flow analyses, introduced in \autoref{ch:confidentialityanalysis}.
Last, we revisited the relation of \acp{ADD} and uncertainty and existing approaches to uncertainty management.

In the second part of this chapter, we investigated work regarding software architecture and confidentiality.
Here, many model-based security analyses and data flow analyses have been proposed.
They share common approaches like design time modeling and analysis or using propagation to better assess the system's security.
In the last part of this chapter, we discussed work that considers confidentiality under uncertainty, which yielded the smallest amount of publications.
Here, we summarized existing uncertainty-aware confidentiality analyses and access control under uncertainty.

In sum, we provided a comprehensive yet comprehensible overview of the state of the art.
By comparing our results from the individual sections, we can derive two initial findings on related work.
First, related work often covers two, but not all three aspects of our approaches: Architectural uncertainty-aware analysis do not focus on confidentiality, architecture-based confidentiality analysis do not consider uncertainty, and uncertainty-aware confidentiality analysis lack the architectural abstraction.
Second, our impression of related work supports the results from the aforementioned surveys \cite{troya_uncertainty_2021,hezavehi_uncertainty_2021}, e.g., the lack of supporting heterogeneous uncertainty, the lack of consolidated notions of uncertainty and providing tool support, or the challenge of end-to-end approaches \cite{weyns_introduction_2020}.

However, recent advances like the \ac{OMG} \ac{PSUM} standardization process \cite{PSUM}, or the research agenda to understand and manage uncertainty \cite{weyns_towards_2023} show that the community is well aware of these challenges and actively addresses them.
We hope that the contributions of this thesis, which were designed with these challenges in mind, also contribute to this research.
Ultimately, please note that this related work chapter does not present a \ac{SLR}.
Thus, it is possible---and even probable---that we missed some related publications.

This chapter summarized work related to our contributions.
These contributions are presented in \readingpath{ch:classification}, \readingpath{ch:impactanalysis}, and \readingpath{ch:confidentialityanalysis}.
To learn more about the foundations, see \readingpath{ch:foundations}.
Next, we will conclude this thesis in \readingpath{ch:conclusion}.





\section{In Simpler Words}%
\label{sec:relatedwork:simple}

Our research is part of three major research communities.
First, the software architecture community researches how we can design better architectures to enhance the quality of software systems.
Second, the \acfp{SAS} community researches how we can build more flexible systems that adapt themselves when facing uncertainty.
Third, the security community researches how to analyze and ensure security-related properties like confidentiality.
In all three communities, many ideas are related to our work.
In this chapter, we give an overview of the most related publications.

First, we look into architecture-based approaches that also consider uncertainty.
Here, many researchers proposed methods to optimize software architectures regarding uncertainty and to help software architects make better decisions when designing software systems under uncertainty.
We learned a lot from these approaches and tried to address the identified shortcomings in our contributions.
Second, we look into architecture-based approaches to analyze confidentiality.
Here, some of the most related work exists, e.g., the data flow analysis that inspired our data flow analysis framework, presented in \autoref{ch:confidentialityanalysis}.
However, these approaches do not consider uncertainty and thus do not solve the problem we want to address with our work.
Third, we look into work that tries to understand confidentiality while considering uncertainty.
Here, we only found a few related publications, some of which are in a preliminary state.
In sum, we did not find a single publication that does exactly what we propose in this thesis---which is a good thing, because otherwise, the novelty of our work would be limited.
