\chapter{Conclusion}%
\label{ch:conclusion}%
 

This chapter concludes this dissertation.
First, we look back and give a short summary on the contributions, their validation, and the key findings of this thesis.
Afterward, we present the key benefits of our work.
Last, we comprehensively discuss future work and current and future research directions.
Note that all previous chapters comprise detailed conclusion and outlook chapters that are summarized here.
This is especially true for the contribution chapters, where we also thoroughly discussed the assumptions and limitations of our contributions.
For detailed summaries, please refer to \autoref{sec:classification:summary} regarding our classification and catalog approach (\C{1}), to \autoref{sec:impactanalysis:summary} for our uncertainty impact analysis and propagation (\C{2}), to \autoref{sec:confidentialityanalysis:summary} for our approaches to uncertainty-aware confidentiality analysis (\C{3}), and to \autoref{sec:evaluation:summary} regarding the evaluation.





\section{Summary}%
\label{sec:conclusion:summary}

This thesis was located in the intersection of three topics: confidentiality, software architecture, and uncertainty.
Confidentiality demands that information is not disclosed to unauthorized persons or organizations \cite{international_organization_for_standardization_isoiec_2018}, which is a crucial quality property due to the high level of connection and the growing volume of data in modern software systems \cite{olivero_security_2019}.
To address the high number of data breaches \cite{cheng_enterprise_2017,gatzlaff_effect_2010,ayyagari_exploratory_2012}, confidentiality should be considered early in the design.
However, existing software architecture-based confidentiality analyses \cite{seifermann_architectural_2022,walter_context-based_2023,pilipchuk_architectural_2021,peldszus_secure_2019,boltz_extensible_2024} do not consider uncertainty, i.e., the lack of information or knowledge regarding the software systems or its environment \cite{international_organization_for_standardization_isoiec_2018,perez-palacin_uncertainties_2014,acosta_uncertainty_2022}.
This major challenge is also acknowledged in related work, e.g., as \enquote{lack of systematic approaches for managing uncertainty} \cite{hezavehi_uncertainty_2021} or as the need of software engineers \enquote{to identify the types of uncertainty that can affect their application domains} \cite{troya_uncertainty_2021}.
Put simply, software engineering is complex \cite{weyns_introduction_2020} and uncertainty is uncertain \cite{garlan_unknown_2021}.

To address this, we introduced an approach that combines the analysis of uncertainty with the analysis of confidentiality.
Based on the classification of uncertainty regarding confidentiality (\C{1}), we presented two central analysis approaches.
First, an uncertainty impact analysis (\C{2}) that propagates the effects of uncertainty within the architectural model to predict the potential impact.
Second, uncertainty-aware data flow analyses (\C{3}) that extend the concept of data flow-based confidentiality analysis \cite{seifermann_data-driven_2019,seifermann_detecting_2022,seifermann_architectural_2022,boltz_extensible_2024} by uncertainty awareness.
By embracing uncertainty as a first-class concern within software architecture \cite{garlan_software_2010}, software architects become aware of the consequences and can identify confidentiality violations due to uncertainty.
To address the lack of tool-assisted approaches to analyzing and managing uncertainty \cite{troya_uncertainty_2021,hezavehi_uncertainty_2021}, we provided tool-support for the identification (\arcen), propagation (\uia), and uncertainty-aware confidentiality analysis (\abunai).
This addresses the challenges \cite{hahner_dealing_2021} of understanding the relation of confidentiality and uncertainty, representing uncertainty in architectural abstraction regarding confidentiality, and analyzing confidentiality under uncertainty.
We used a running example throughout this thesis to exemplify the ideas and new concepts.
Based on the scenario of a simplified online shop, where customers browse for available items and make purchases, we introduced uncertainty sources like user behavior, data processing, deployment, or provider trustworthiness.
To deal with uncertainty, we target four key activities: identification, classification, propagation, and analysis.
The resulting procedure is aligned with our three contributions and our tool support, see \autoref{ch:overview}.
In the following, we briefly summarize the three contributions of this dissertation:


\paragraph{Identification and classification of uncertainty (\C{1})}
Our first contribution comprised a classification of uncertainty regarding its impact on confidentiality \cite{hahner_classification_2023} and a catalog approach to support the identification of new uncertainty sources.
We detailed the relation of uncertainty sources and their impact and investigated existing uncertainty taxonomies \cite{walker_defining_2003,perez-palacin_uncertainties_2014,ramirez_taxonomy_2012,mahdavi-hezavehi_classification_2017,bures_capturing_2020}.
Here, we also coined the term \emph{software-architectural} uncertainty that describes uncertainty on architectural abstraction where early awareness helps in the assessment and mitigation.
Afterward, we defined a classification tailored to confidentiality on architectural abstraction.
The central category is the \emph{Architectural Element Type}, with options like \emph{External} uncertainty, or \emph{Actor} uncertainty, that are used throughout the remaining thesis.
We used this classification to represent uncertainty as a first-class concern \cite{garlan_software_2010} within \acfp{DFD}.
Last, we presented an approach to the collaborative identification of uncertainty sources \cite{hahner_arcn_2024}, which results in the tool-supported catalog \arcen.
Combined, the classification and the catalog approach contribute to our understanding of the relation between uncertainty and confidentiality in software architecture and provide an answer to \RQ{1}.

\finding{An uncertainty classification tailored to confidentiality provides the terminology to describe uncertainty sources and their impact.
The integration of this classification into an uncertainty source catalog simplifies the identification of new uncertainty sources and improves the understanding of software architects.
}


\paragraph{Uncertainty impact analysis (\C{2})}
Our second contribution focused on the propagation of uncertainty within architectural models to define an uncertainty impact analysis.
First, we discussed the representation of uncertainty in architectural models \cite{troya_uncertainty_2021,hahner_architecture-based_2023}.
Afterward, we provided algorithms for the propagation of all five uncertainty types according to our classification within architectural models and also within \acp{DFD}.
Combined, this enabled the definition of an uncertainty impact analysis that predicts the potential impact of uncertainty sources and potential locations of confidentiality violations.
With \uia, we also provided tool support for architecture-based uncertainty impact analysis.
By combining the previously introduced catalog approach with this impact analysis, we also addressed the identification problem in this analysis.
Last, we generalized our findings \cite{camara_uncertainty_2024} to address the \acf{UIP} by defining the notion of \acf{UFD}.
To conclude, this answers \RQ{2} about the propagation and impact assessment of uncertainty regarding confidentiality.

\finding{The propagation of uncertainty within architectural models helps to estimate the potential impact and to predict confidentiality violations.
Moreover, uncertainty propagation can also be used to analyze uncertainty interactions.}


\paragraph{Uncertainty-aware confidentiality analysis (\C{3})}
As the third contribution, we presented four approaches to uncertainty-aware data flow analysis to analyze the confidentiality of architectural models.
First, we introduced an extensible data flow analysis framework \cite{boltz_extensible_2024} and provided the conceptual basis of uncertainty-awareness in data flow analysis.
Afterward, we presented four analysis approaches that differ in the complexity of the analysis algorithm and in supported uncertainty types.
We presented two type-specific approaches, tailored to structural uncertainty, and environmental uncertainty, and two type-agnostic analyses.
We also introduced \abunai, which provides tool support for the most advanced type-agnostic approach.
Last, we also discussed the complexity of uncertainty-aware data flow analysis as naive approaches suffer from the combinatorial explosion, known from design space exploration \cite{koziolek_automated_2011}.
These uncertainty-aware data flow analyses represent an answer to \RQ{3}, which asked about how to analyze confidentiality requirements while considering uncertainty.

\finding{Architecture-based confidentiality analysis can be achieved by considering the impact of uncertainty on the flowing data.
Here, building on the findings of uncertainty propagation and interaction helps to ensure accuracy and scalability while supporting all identified uncertainty types.}


The validation of our contributions was based on a \acf{GQM} plan \cite{basili_goal_1994,basili_methodology_1984} that comprised 8 goals, 19 questions, and 32 metrics.
First, we evaluated our uncertainty classification according to an taxonomy evaluation method \cite{kaplan_introducing_2022}, and the usability of our catalog approach.
Afterward, we evaluated the uncertainty impact analysis using the same evaluation plan already used to evaluate change impact analysis \cite{rostami_architecture-based_2015,rostami_architecture-based_2017}, i.e., focusing on accuracy and effort reduction.
Last, we evaluated the scalability and accuracy of our uncertainty-aware data flow analyses.
As part of the evaluation, we used six evaluation scenarios, see \autoref{ch:evaluationscenarios}.
This scenarios originate from related work \cite{katkalov_model-driven_2013,katkalov_modellgetriebener_2017,seifermann_data-driven_2019,rausch_common_2008,leinweber_leveraging_2023} or real-world software systems \cite{robert_koch_institute_open-source_2020,enaya_case_2024,prechelt_finding_2002,saglam_obfuscation-resilient_2024}.
All scenarios have different confidentiality requirements, and different confidentiality violations due to uncertainty.
Additionally, we conducted two user studies.
Overall, we find the evaluation results satisfying.
Many measurements represent perfect results, while others show room for improvement despite being acceptable.
The biggest room for improvement is in the usability of the classification and catalog approaches, which is not surprising as both represent novel approaches with only prototypical tool support.
However, we especially want to highlight the impressive effort reduction of 86\% provided by our uncertainty impact analysis and the complexity reduction of our uncertainty impact-aware data flow analysis, see \autoref{sec:evaluation:summary}.
In \autoref{ch:introduction}, we defined the research goal of this dissertation:

\ResearchGoal

We conclude, that our three Contributions \C{1} -- \C{3} address this research goal and the evaluation results show that our contributions are of high quality.
Moreover, the contributions address other gaps known from recent surveys, such as the need for systematic approaches for managing uncertainty \cite{hezavehi_uncertainty_2021}, the representation of uncertainty in models \cite{troya_uncertainty_2021}, and the need for uncertainty-aware end-to-end approaches \cite{weyns_towards_2023}.
Last, some of the concepts presented in this thesis have already been generalized, e.g., to address uncertainty interactions \cite{camara_uncertainty_2024}.
We hope that our findings help the research community to further advance in the areas of uncertainty, confidentiality, and software architecture.





\section{Benefits}%
\label{sec:conclusion:benefits}

We see many benefits in the contributions of this dissertation.
These benefits were already enumerated in the summaries of the contributions chapters, i.e., in \autoref{sec:classification:summary}, \autoref{sec:impactanalysis:summary}, and \autoref{sec:confidentialityanalysis:summary}.
In the following, we repeat the key benefits.

The explicit modeling of uncertainty and confidentiality on architectural abstraction and the incorporation of these concepts in architecture-based confidentiality analysis grants several benefits.
The benefits of our classification include precise terminology to discuss and understand uncertainty regarding confidentiality. 
This supports both software architects and security experts in modeling and analyzing software systems. 
As proposed in \autoref{ch:overview}, we do not require an additional \emph{uncertainty expert} role, as the required knowledge is contained in the classification, the uncertainty source catalog, and all analyses. 
Here, our tooling \arcen represents a good starting point.
By identifying and assessing uncertainty sources early, the reasoning and prioritization of \acfp{ADD} is simplified and costly backtracking is minimized. 
This is especially true regarding uncertainty interactions \cite{camara_addressing_2022}, which represent uncertainty impacts that are particularly hard to find and mitigate.
Last, our classification lays the foundations for further integration of uncertainty in the architecture-based confidentiality analyses.

Propagating uncertainty helps software architects in handling uncertainty \cite{acosta_uncertainty_2022}.
Architecture models can be annotated with uncertainty sources from existing catalogs \cite{hahner_arcn_2024,hahner_classification_2023}, which helps in the documentation and to raise awareness. 
The analysis helps in predicting and mitigating confidentiality violations. 
Using a confidentiality analysis for this purpose would require software architects to manually understand and model the impact of uncertainty, which requires more effort and expertise.
As proposed in our procedure in \autoref{sec:overview:procedure}, we see uncertainty propagation as the first step to analyzing an architectural model prior to any detailed confidentiality analysis.
Here, our tooling \uia presents a good starting point.
The calculated models of our analysis can also be used for regression testing or to handle uncertainty at runtime \cite{derakhshanmanesh_model-integrating_2019}.

Our uncertainty-aware data flow analysis approaches enable an earlier and more accurate assessment of the overall system's confidentiality.
Detecting and repairing confidentiality issues earlier can reduce cost \cite{boehm_software_2001}.
Here, we do not even require a fully defined software architecture as the modeled uncertainty can be reused compared to regression tests.
The results of our uncertainty-aware confidentiality analysis utilize the available information within architectural models more efficiently and shall enable an easier interpretation by software architects. 
This might also minimize change efforts, subsequent faults, and costs due to confidentiality violations.
This is especially true for large models with specific confidentiality violations that have to be traced through the complete system to become manageable. 
Here, we refer to our tooling \abunai.
Including uncertainty in the design process may increase the flexibility at runtime when facing unexpected context changes \cite{bures_capturing_2020}.
This shall help in building more resilient software systems.
By including the impact of uncertainty, the quality of confidentiality analysis results increases due to higher coverage of considered problems and possible violations.
This is especially true for runtime uncertainty and related to topics like \acf{SAS} \cite{sobhy_evaluation_2021,hezavehi_uncertainty_2021} and antifragility \cite{gorgeon_anti-fragile_2015,de_bruijn_antifragility_2020,ramezani_approaches_2020}.
Lastly, our results can also be used as guidelines.





\section{Future Work}%
\label{sec:conclusion:futurework}

In the following, we provide an overview of potential future work.
Note that we make no claim to completeness, as there is still a lot to do in this area \cite{weyns_towards_2023,hezavehi_uncertainty_2021}.
For the sake of traceability in future publications, we enumerate all 55 ideas for future work as \FW{1} -- \FW{55}.
We divide future work into four parts: further evaluation of our approach, extension of our approach and its tool support, integration of our analyses with other approaches or frameworks, and generalization of our contributions.


\paragraph{Evaluation}
Although we already presented a comprehensive evaluation in \autoref{ch:evaluation}, further evaluating our approach is always possible.
For instance, by applying our uncertainty impact analysis and uncertainty-aware data flow analyses to more evaluation scenarios from other domains (\FW{1}).
Furthermore, we could repeat the user studies with the final versions of both the uncertainty classification (\FW{2}) and the uncertainty catalog approach (\FW{3}) to investigate changes in usability.
Especially regarding the catalog approach, a user study that compares the performance to a baseline without tool support would be expedient (\FW{4}).
Future work could revisit the validity of identified confidentiality violations compared to previous approaches \cite{seifermann_data-driven_2019,seifermann_detecting_2022}, e.g., by utilizing formal methods (\FW{5}).
Here, also the formal representation of \acfp{NDFD} beyond \acfp{DAG} can be expedient (\FW{6}), comparable to the graph-based formalization of \textcite{alshareef_precise_2022}.
Last, broader studies could evaluate the usability of our analyses in real-world software architectures (\FW{7}).


\paragraph{Extension}
During this dissertation, we identified several extension possibilities for all contributions.
First, our catalog approach and its tool support \arcen could be extended, e.g., by integrating recommendation techniques (\FW{8}).
Also, the explainability \cite{bersani_conceptual_2023,bersani_architecting_2023} of uncertainty is relevant (\FW{9}), as the understandability can impact the usability of software architects.
This does not only include recommending and explaining uncertainty sources in isolation but also considering the software architecture under study (\FW{10}).
Here, the application of \acfp{LLM} can be expedient (\FW{11}).

Other extensions could target the beginning and the ending of our approach, i.e., the requirement phase and the mitigation of uncertainty \cite{niehues_architecture-based_2025}.
First, we could (semi-) automatically derive detailed confidentiality requirements based on more abstract protection goals (\FW{12}).
This includes the extension of the modeling approach of confidentiality requirements as data flow constraints \cite{hahner_modeling_2021}, e.g., to express multiple levels (\FW{13}).
An example would be higher-level requirements that originate from the law and lower-level requirements that represent, e.g., access control policies (\FW{14}).
This could result in approaches to uncertainty-aware access control policy refinement \cite{hahner_architectural_2021} (\FW{15}).
There is more research required at the intersection of legal sciences and software architecture \cite{boltz_model-based_2022}, e.g., to better understand and support the legal analysis of software architecture and the collaboration of experts from both fields (\FW{16}).
Furthermore, defining patterns of confidentiality requirements and data flow constraints could increase the usability of architecture-based data flow analysis (\FW{17}).
All constraint-related future work would also benefit from a clearly defined meta model based on the new data flow analysis framework (\FW{18}) that extends previous work \cite{hahner_modeling_2021}.
Last, we did not consider the analysis of uncertainty within requirements (\FW{19}), which can highly impact the analysis results.

Our uncertainty-aware data flow analyses yield confidentiality violations due to uncertainty.
Future work could extend this to provide means to mitigate the effects of uncertainty (\FW{20}).
Examples are the automated repair of models to comply with confidentiality requirements (\FW{21}), or constraint optimization (\FW{22}).
Also the holistic automated repair of confidentiality-violating models without even considering uncertainty could be possible (\FW{23}).
Regarding the propagation of uncertainty with \uia, the precision of our impact analysis could be increased by further minimizing the impact set.
Currently, starting from an initial uncertainty impact location, the remaining data flows are added to the impact set.
However, based on guarantees or checkpoints within the system \cite{camara_software_2020}, e.g., guaranteed encryption, the impact set could be reused (\FW{24}).
Last, future work could address the \abunai tool support.
Here, the modeling of uncertainty sources could be simplified (\FW{25}), e.g., by providing graphical editors that surpass the quality of the diagrams shown in \autoref{sec:appendix:runningexample} (\FW{26}), or a \acf{DSL} under uncertainty (\FW{27}).
Also, the integration of low-code approaches in the data flow analysis and \abunai is possible (\FW{28}).
Here, the textual representation of \acf{PCM} instances and confidentiality-related information like labels and assignments could be expedient (\FW{29}).
This would also address the need for end-to-end approaches \cite{weyns_towards_2023}.
Last, our work would also benefit from advances in the underlying data flow analysis framework \cite{boltz_extensible_2024,huller_towards_2024, niehues_integrating_2024}, e.g., the graphical display of confidentiality violations (\FW{30}) and uncertainty (\FW{31}), support for cyclic data flow diagrams \cite{arp_analyzing_2024} (\FW{32}), and graphical editors to annotate uncertainty sources to the model (\FW{33}).


\paragraph{Integration}
Besides extending our approach, future work can also address the previously discussed integration with other analysis approaches or into other analysis frameworks.
First, future work should investigate whether our approach can be combined with CodeQL \cite{github_codeql_2021,de_moor_ql_2008} to enable code-based uncertainty propagation (\FW{34}).
This could also enable the coupling of data flow analysis regarding architectural models and code \cite{reiche_modeling_2021,lochau_model-based_2012} (\FW{35}).
Another potential integration could be the extension of extracted \acp{DFD} from microservices \cite{schneider_automatic_2023} by uncertainty \cite{niehues_integrating_2024} to increase the expressiveness (\FW{36}).
Other viable combinations are the integration of uncertainty in attack path analysis \cite{walter_architecture-based_2023-1} (\FW{37}) or other architecture-based security analyses \cite{katkalov_model-driven_2013,goos_umlsec_2002,lodderstedt_secureuml_2002} (\FW{38}).
Here, natural language processing can be used to derive data flow constraints from requirements (\FW{39}) to enable a continuous security analysis that comprises all development phases \cite{schulz_continuous_2021} (\FW{40}).
Furthermore, based on the software architecture, the uncertainty-aware confidentiality analysis can be combined with other quality properties (\FW{41}), e.g., performance, cost, or reliability.
This could be achieved with an extension of design space exploration approaches like PerOpteryx \cite{koziolek_peropteryx_2011} (\FW{42}), similar to \autoref{sec:confidentialityanalysis:structural}.
Also, the coupling of our uncertainty catalog approach \arcen with other catalogs is possible (\FW{43}), e.g., by relating uncertainty to security solutions \cite{jasser_constraining_2019}.
Last, our analyses could be integrated into already existing frameworks (\FW{44}), e.g., RADAR \cite{busari_modelling_2019,busari_radar_2017}, Rainbow \cite{garlan_rainbow_2004}, DeTum \cite{famelis_managing_2019}, or the approach of \textcite{lytra_supporting_2013}.


\paragraph{Generalization}
Last, we present potential future work to generalize our findings.
First, we only considered simple uncertainty interaction between secondary uncertainty and primary uncertainty.
Here, our findings could be used to also address other types of uncertainty interactions (\FW{45}), see \autoref{sec:confidentialityanalysis:abunai}.
Our findings could also serve the general discussion about the \ac{UIP} \cite{camara_addressing_2022,camara_uncertainty_2022,weyns_towards_2023} (\FW{46}), as initially shown by \textcite{camara_uncertainty_2024}.
The same applies to the upcoming discussion about antifragility \cite{grassi_towards_2023,gorgeon_anti-fragile_2015,de_bruijn_antifragility_2020,grassi_tao_2021,ramezani_approaches_2020,burton_resilience_2024}.
Here, especially uncertainty propagation could become valuable (\FW{47}).
In \acp{SAS}, future work could consider the application of our analyses at runtime (\FW{48}).
Uncertainty propagation has also been discussed in the context of coupled models of \acp{CPS} \cite{acosta_uncertainty_2022}.
Here, existing consistency relations between models can be used for the uncertainty propagation beyond a single model.
However, understanding the relation of intra-model and inter-model uncertainty propagation requires additional research (\FW{49}).
Furthermore, it should be researched which uncertainty types and model elements are applicable for both (\FW{50}) and also regarding other quality properties than confidentiality (\FW{51}). 
This also raises the question for future work whether the viable combination of a classification and a catalog, as presented with our first contribution, is generalizable (\FW{52}).
We presented approaches to represent uncertainty within architectural models, \acp{DFD}, and \acp{DAG}.
Future work could define a graphical notation of uncertainty in \acp{DFD} to enhance documentation and communication (\FW{53}).
An initial proposal is shown in \autoref{sec:appendix:ndfd}.
The easiest generalization could be the consideration of related quality properties that can also be analyzed using \acp{DFD} \cite{boltz_extensible_2024}, e.g., integrity (\FW{54}).
Last, future work could define design patterns for uncertainty-resilient software architectures (\FW{55}).

In sum, future work has many directions, from revisiting the evaluation, to integrating, extending and generalizing the contributions of this dissertation.
All of this future work will build on our research, which was the investigation of architecture-based confidentiality analysis under uncertainty. 
We look forward to future insights and research results!





\section{In Simpler Words}%
\label{sec:conclusion:simple}

This chapter concludes this dissertation.
First, we summarize the content of this thesis, thereby focusing on our three contributions.
The first contribution was a classification of uncertainty regarding confidentiality.
Here, we provided the foundations to reason about uncertainty.
For instance, we discussed which types of uncertainty are relevant regarding confidentiality, and which are not.
Furthermore, we presented a catalog of uncertainty sources that helps software architects identify new uncertainty sources in the software architecture they try to design or analyze.
An example of such an uncertainty source is user behavior, which is always unpredictable to some degree.
The second contribution was an uncertainty impact analysis that propagates uncertainty within the software architecture to predict confidentiality violations.
This helps software architects to quickly assess the potential uncertainty impact without much additional effort.
For instance, they could get the result that an uncertainty only affects non-critical parts of the software, which does not violate confidentiality.
The third contribution comprised four uncertainty-aware data flow analyses to identify confidentiality violations due to uncertainty.
This helps software architects to build more resilient software systems that ensure confidentiality even when facing uncertainty.
Using our approach requires less expertise as the knowledge is encoded as part of the analyses.
Due to the automation of our analysis, software architects also require less effort when analyzing an architectural model.
Last, our approach helps in the documentation and communication of uncertainty in architectural models and diagrams.

We conclude this chapter by presenting 55 ideas for future work.
Of course, there is much more research to conduct in this area, but our ideas can serve as a starting point.
We discuss future work regarding further evaluation of our contributions, and many analysis extensions that also add features to our tooling.
Afterward, we show how our concepts can be integrated into other work, e.g., existing frameworks for architecture-based analysis.
Overall, such future work would provide better and more comprehensive tools for software architects and further simplify their work---besides helping to answers current open research questions.
Last, we discuss the generalization of our findings.
Here, especially uncertainty propagation represents a promising concept for other applications.
For instance, it already has been discussed in the context of uncertainty interactions and in the propagation between different models of the same software system.
After four years of research, this concludes our work.
We hope that our results will contribute to advancing research in the area of software architecture, confidentiality, and uncertainty\footnote{For you, dear reader, who certainly likes numbers as much as I do: In this dissertation, the word \emph{uncertainty} was used 3393 times. You are welcome.}.
