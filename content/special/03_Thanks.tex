\selectlanguage{ngerman}
\Abstract[Danksagung]{%
\emph{1560 Tage.} 

1560 Tage liegen zwischen meinem ersten Arbeitstag als wissenschaftlicher Mitarbeiter am Karlsruher Institut für Technologie (KIT) und der erfolgreichen Verteidigung meiner Dissertation.
Zum Glück musste ich den Weg an den allermeisten dieser Tage nicht allein gehen---wofür ich mich nachfolgend bedanken möchte.

Zuallererst und vor allem möchte ich mich bei meiner Frau Alina bedanken, die mich stets unterstützt hat, egal, ob ich gerade übermotiviert und viel zu fokussiert geforscht oder verzweifelt alles infrage gestellt habe.
Sie hat mir den bei einer Promotion dringend notwendigen Rückhalt gegeben, jederzeit an mich geglaubt und somit diesen großen Erfolg überhaupt erst möglich gemacht.
Ebenfalls möchte ich meinen Eltern von ganzem Herzen für ihre Unterstützung und unermüdliche Ermutigung danken.
Bedanken möchte ich mich auch bei meinen Freunden und Kollegen außerhalb der Universität, denen ich auch in den letzten vier Jahren nicht die Aufmerksamkeit geschenkt habe, die sie verdient hätten.

Während meiner Zeit am Lehrstuhl hatte ich die Möglichkeit mit vielen motivierten und äußerst fähigen Studierenden zusammen zu arbeiten.
Hier möchte ich mich insbesondere bei Niko bedanken, mit dem ich als Allererstes an meinen neuen Ideen geforscht habe, und bei Felix, der mir am Schluss tatkräftig geholfen hat, die prototypische Umsetzung meiner Konzepte zu finalisieren.
Ein weiteres Danke für größere und kleinere Beiträge zu meiner Forschung geht an Gabriel, Tizian, Denis, Daniel, Oliver, Anne-Kathrin, Alexander, Tom, Lilian, Benjamin und Jonas.
Ich hatte während meiner Promotion zahlreiche Gelegenheiten zu reisen und meinen Horizont zu erweitern.
Ich möchte mich insbesondere bei Simon, Marc und Mario für die neuen Einblicke bedanken.
Ebenfalls bedanken möchte ich mich bei den Organisatoren und Teilnehmern des Bertinoro USAS Seminars und der Dagstuhl Seminare, welche mir besonders in Erinnerung geblieben sind.

Natürlich dürfen in dieser Danksagung meine großartigen Kolleginnen und Kollegen am Lehrstuhl nicht fehlen.
Zuerst möchte ich mich bei Stephan und Maximilian bedanken, die nicht nur wichtige Vorarbeit für meine Forschung geleistet haben, sondern mich auch herzlich am Lehrstuhl aufgenommen und insbesondere in der Anfangszeit angeleitet haben.
Ein weiterer Dank gebührt meinen Bürokollegen Frederik, Sandro, Nicolas und Nils für den regen---und oft genug nicht fachbezogenen---Austausch.
Insbesondere bei Nicolas möchte ich mich für die jahrelange extrem erfolgreiche und spaßige Zusammenarbeit bedanken: Wie wir innerhalb so kurzer Zeit von Grund auf ein neues Analyse-Framework samt Team aufgebaut haben, war mindestens genauso ein Highlight wie die vielen gemeinsamen Kaffeepausen und Weinwanderungen.
Auch bei Timur und Larissa möchte ich mich bedanken: Zu dritt haben wir nicht nur fast zeitgleich mit der Promotion begonnen, sondern diese auch fast gleichzeitig abgeschlossen; der regelmäßige Austausch im alltäglichen Wahnsinn hat sehr gutgetan.
Zudem habe ich mit Timur---und später Dominik und Robin---zusammen den Plagiatsdetektor JPlag wiederbelebt und von einem Software-Archäologieprojekt in ein erfolgreiches und aktives Open-Source-Projekt verwandelt.
Das war eine steile Entwicklung und ich bin extrem stolz darauf, was wir daraus gemacht haben.
Steil waren auch die Wände, die ich während meiner Promotion erklommen habe---nicht nur im übertragenen Sinne.
Hier möchte ich Tobias danken, der mich nicht nur zu einem neuen Hobby geführt, sondern auch während des Diss-Endspurts unterstützt hat.

Ein weiteres Danke gebührt den übrigen Lehrstuhl-Seniors.
Hier möchte ich zuerst Anne, Christopher und Erik für die zahlreichen Hinweise, Hilfestellungen und Möglichkeiten danken.
Ebenfalls bedanken möchte ich mich bei Raffaela, welche viele Grundlagen für meine Arbeit erforscht und mir zahlreiche Türen geöffnet hat.
Der meiste Dank gebührt hier aber natürlich Robert, welcher mich von Anfang an in meiner Promotion begleitet und betreut hat.
Die zahlreichen Diskussionen, die Einbindung in Forschungsprojekte sowie die gemeinsame Publikationsplanung haben eine maßgebliche Auswirkung auf meine Forschung gehabt.
Zum Schluss---denn Professoren stehen traditionell am Ende der Autorenliste---möchte ich mich bei meinem Erstgutachter und Betreuer Ralf bedanken.
Seine Anregungen, Feedback, und manchmal auch herausfordernden Ideen haben diese Doktorarbeit deutlich beeinflusst.
Noch mehr bin ich aber für alles außerhalb der Promotion dankbar; seien es die Einblicke in die Wissenschaft, die zahlreichen Möglichkeiten, mich kreativ einzubringen, oder die stundenlangen Ausflüge in die Philosophie.

Mein Dank gilt auch allen, die mich während meiner Promotion begleitet haben, aber hier nicht direkt oder indirekt genannt wurden.
Diese 1560 Tage waren eine herausfordernde Zeit, in der ich extrem viel erleben durfte.
Dennoch bin ich froh, dieses Kapitel abgeschlossen zu haben und hinter mir lassen zu können.
Danke für alles. Auf die Zukunft!
%
}